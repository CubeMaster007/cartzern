\documentclass[showpacs,%
  nofootinbib,aps,superscriptaddress,%
  eqsecnum,prd,notitlepage,showkeys,10pt]{article}

\addtolength{\oddsidemargin}{-1in}
\addtolength{\evensidemargin}{-1in}
\addtolength{\textwidth}{2in}

\usepackage{amssymb}
\usepackage{amsmath}
\usepackage{graphicx}
\usepackage{dcolumn}
\usepackage{hyperref}

\newcommand{\sincos}{\resizebox!{1em}{\(\begin{array}c\cos\vspace{-1ex}\\\sin\end{array}\)}\hspace{-.5ex}}

\begin{document}

\title{Cartesian Zernike Polynomials}
\author{Adam Mendenhall}

\begin{abstract}
We derive formulae for the Zernike polynomials with cartesian coordinate arguments, without the use of trigenometric functions.  In many computing applications, the input coordinates are cartesian to begin with.  In these circumstances it is very undesirable to compute the polar representation and then evaluate the Zernike polynomial.
\end{abstract}

\maketitle

\section{Introduction}
\label{sec:intro}

The Zernike polynomials are a set of scalar valued functions defined on the unit circle.  The set is orthogonal with respect to the standard metric
\[\langle f,g\rangle=\frac1\pi\int_0^1\int_0^{2\pi}f(r,\theta)g(r,\theta)\;\mathrm d\theta\mathrm dr.\]
Most indexing schemes define a unique polynomial for each index tuple $(n,m)$ with $n\in\mathbb N$ and $m\in\mathbb Z$ with $|m|\leq n$ and $n\equiv m\pmod2$.  For a given $(n,m)$, we define the Zernike polynomial
\[Z_n^m(r,\theta)=\left\{\begin{array}{cl}R_n^m(r)\cos m\theta&\text{if }m>0\\R_n^{-m}(r)\sin m\theta&\text{if }m<0\\R_n^0(r)&\text{if }m=0\end{array}\right.\]
where the radial polynomials $R_n^m$ are defined for $m\geq0$ as 
\[R_n^m(r)=\sum_{k=0}^{\frac{n-m}2}(-1)^k\binom{n-k}k\binom{n-2k}{\frac{n-m}2-k}r^{n-2k}.\]
The radial polynomials are also defined recursively with $R_n^n(r)=r^n$ and $R_{n+2}^n(r)=\left((n+2)r^2-(n+1)\right)r^n$, otherwise
\[R_n^m(r)=-\frac{n(n+m-2)(n-m-2)}{(n+m)(n-m)(n-2)}R_{n-4}^m(r)+\frac{2(n-1)(2n(n-2)r^2-m^2-n(n-2))}{(n+m)(n-m)(n-2)}R_{n-2}^m(r).\]
%\[\cos((n+2)\theta)=2\cos((n+1)\theta)\cos\theta-\cos(n\theta),\]
%\[\sin((n+2)\theta)=2\sin((n+1)\theta)\cos\theta-\sin(n\theta).\]
We will use expressions for $\sincos\big(n\theta\big)$ as polynomials in $\cos\theta$ to derive explicit and recursive formulae for $Z_n^m$ as polynomials in $x$ and $y$.

\section{Main Dish}
\label{sec:maindish}

Let $x=r\cos\theta$ and $y=r\sin\theta$, we have $r=\sqrt{x^2+y^2}$.

\subsection{Positive $\boldsymbol m$}
\label{sec:pm}
Take the formula due to Chebyshev
\[\cos(m\theta)=\frac m2\sum_{k=0}^{\left\lfloor\frac m2\right\rfloor}\frac{(-1)^k}{m-k}\binom{m-k}k(2\cos\theta)^{m-2k}\]
and the radial Zernike polynomial formula to define for $m>0$
\[Z_n^m(x,y)=\frac m2\sum_{\zeta=0}^{\frac{n-m}2}\sum_{\xi=0}^{\left\lfloor\frac m2\right\rfloor}\frac{(-1)^{\zeta+\xi}}{m-\xi}\binom{m-\xi}\xi\binom{n-\zeta}\zeta\binom{n-2\zeta}{\frac{n-m}2-\zeta}(2x)^{m-2\xi}\left(x^2+y^2\right)^{\xi-\zeta+\frac{n-m}2}\]
By binomial expansion, we have $Z_n^m(x,y)=$
\[\frac m2\sum_{\zeta=0}^{\frac{n-m}2}\sum_{\xi=0}^{\left\lfloor\frac m2\right\rfloor}\sum_{\varsigma=0}^{\xi-\zeta+\frac{n-m}2}2^{m-2\xi}\frac{(-1)^{\zeta+\xi}}{m-\xi}\binom{m-\xi}\xi\binom{n-\zeta}\zeta\binom{n-2\zeta}{\frac{n-m}2-\zeta}\binom{\xi-\zeta+\frac{n-m}2}\varsigma x^{n-2\zeta-2\varsigma}y^{2\varsigma}\]
We perform the substitution
\[\begin{array}{c|c}\zeta'=\zeta+\varsigma&0\leq\zeta'\leq\left\lfloor\frac n2\right\rfloor\\\xi'=\xi&\max\left(0,\zeta'-\frac{n-m}2\right)\leq\xi'\leq\left\lfloor\frac m2\right\rfloor\\\varsigma'=\varsigma&\max\left(0,\zeta'-\frac{n-m}2\right)\leq\varsigma'\leq\zeta'\end{array}\]
to achieve (variable domains not repeated for brevity) $Z_n^m(x,y)=$
\[\frac m2\sum_{\zeta'}\sum_{\varsigma'}\sum_{\xi'}2^{m-2\xi'}\frac{(-1)^{\zeta'-\varsigma'+\xi'}}{m-\xi'}\binom{m-\xi'}{\xi'}\binom{n-\zeta'+\varsigma'}{\zeta'-\varsigma'}\binom{n-2\zeta'+2\varsigma'}{\frac{n-m}2-\zeta'+\varsigma'}\binom{\xi'-\zeta'+\varsigma'+\frac{n-m}2}{\varsigma'}x^{n-2\zeta'}y^{2\varsigma'}\]
The diligent reader will identify a bijection between the sets $\{(\zeta,\xi,\varsigma)\}$ and $\{(\zeta'-\varsigma',\xi',\varsigma')\}$, keeping in mind $\left\lfloor\frac n2\right\rfloor=\frac{n-m}2+\left\lfloor\frac m2\right\rfloor$.

It is clear that for $Z_n^{m>0}(x,y)$, the powers of $x$ are $X=[0,n]\cap\{x\equiv n\pmod2\}$ while the powers of $y$ are $Y=[0,n]\cap\{x\equiv0\pmod2\}$ and that the exponents of $Z$ as a polynomial in $x$ and $y$ are $X\times Y\cap\{(x,y)|x+y\leq n\}$.

\subsection{Negative $\boldsymbol m$}
We repeat the process from section \ref{sec:pm} for the Zernike polynomials with negative $m$.  Take the formula due to Chebyshev
\[\sin(m\theta)=\left(\sum_{k=0}^{\left\lfloor\frac{m-1}2\right\rfloor}(-1)^k\binom{m-1-k}k(2\cos\theta)^{m-1-2k}\right)\sin\theta\]
and the radial Zernike polynomial formula to define for $m>0$
\[Z_n^{-m}(x,y)=y\sum_{\zeta=0}^{\frac{n-m}2}\sum_{\xi=0}^{\left\lfloor\frac{m-1}2\right\rfloor}(-1)^{\zeta+\xi}\binom{n-\zeta}\zeta\binom{n-2\zeta}{\frac{n-m}2-\zeta}\binom{m-1-\xi}\xi(2x)^{m-1-2\xi}\left(x^2+y^2\right)^{\xi-\zeta+\frac{n-m}2}\]
By binomial expansion, we have $Z_n^{-m}(x,y)=$
\[y\sum_{\zeta=0}^{\frac{n-m}2}\sum_{\xi=0}^{\left\lfloor\frac{m-1}2\right\rfloor}\sum_{\varsigma=0}^{\xi-\zeta+\frac{n-m}2}2^{m-1-2\xi}(-1)^{\zeta+\xi}\binom{n-\zeta}\zeta\binom{n-2\zeta}{\frac{n-m}2-\zeta}\binom{m-1-\xi}\xi\binom{\frac{n-m}2-\zeta+\xi}\varsigma x^{n-2\zeta-2\varsigma-1}y^{2\varsigma}\]
We perform the substitution
\[\begin{array}{c|c}\zeta'=\zeta+\varsigma&0\leq\zeta'\leq\left\lfloor\frac n2\right\rfloor\\\xi'=\xi&\max\left(0,\zeta'-\frac{n-m}2\right)\leq\xi'\leq\left\lfloor\frac{m-1}2\right\rfloor\\\varsigma'=\varsigma&\max\left(0,\zeta'-\frac{n-m}2\right)\leq\varsigma'\leq\zeta'\end{array}\]
to achieve (variable domains not repeated for brevity) $Z_n^{-m}(x,y)=$
\[y\sum_{\zeta'}\sum_{\varsigma'}\sum_{\xi'}2^{m-1-2\xi'}(-1)^{\zeta'-\varsigma'+\xi'}\binom{n-\zeta'+\varsigma'}{\zeta'-\varsigma'}\binom{n-2\zeta'+2\varsigma'}{\frac{n-m}2-\zeta'+\varsigma'}\binom{m-1-\xi'}{\xi'}\binom{\frac{n-m}2-\zeta'+\varsigma'+\xi'}{\varsigma'}x^{n-2\zeta'-1}y^{2\varsigma'}\]

\end{document}
